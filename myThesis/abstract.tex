\begin{abstract}

Current and historic tactics have leveraged the reduced detectability of \gls{LO} platforms and the confusion of decoys to complicate target identification and prosecution via radar. Ideally, adversary fires will be wasted engaging decoys, while destructive systems continue to advance on their targets. The ability to discern a threatening platform from a decoy will be crucial to increasing the survivability of friendly forces, and enable judicious use of fires. This project examines a potential solution for identifying a target in decoy-saturated airspace. A convolutional Neural Network is trained using RCS data for five targets. Targets are missile-surrogates that shares a similar fuselage, each with their own unique nose-cone. Training data is created by simulating the RCS response of each target using Altair's CADFeko Electromagnetic simulation quite. Data is captured over from 4.5 to 5.5 GHz, over 360 degrees of water-line azimuth angle, random-gaussian noise is applied to the frequency measurements of each angle, and normalized between 0 and 1. Normalized data is duplicated, labeled, and used to train the CNN. The trained network is tested for accuracy using measurement data collected at the Air Force Institute of Technologies compact radar cross section range. Measured targets are 1:1 copies of the simulation targets. The correlation of frequency measurements taken at each angle are correlated between the simulated, and measured targets to identify points of similarity and divergence in the resulting RCS response. These cross-correlation values are used to determine where miss-classifications occur in the CNN.

%Is going to need conclusion data



\end{abstract}
